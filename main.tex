\documentclass[12pt]{article}
\usepackage{amsfonts}\usepackage{amssymb}
\usepackage[french]{babel}
\usepackage{amsmath}
\usepackage[utf8]{inputenc}
\usepackage{calc}
\usepackage{enumerate}
\usepackage{pst-func}
\usepackage{pstricks-add}
\usepackage{hyperref}
\hypersetup{
    pdfauthor={Yoann Morel},
    pdfsubject={Ordre et intervalles},
    pdftitle={Nombres, ordre et intervalles},
    pdfkeywords={mathématiques, cours, exercices, 
      ordre, intervalle} 
}
\hypersetup{
    colorlinks = true,
    linkcolor = red,
    anchorcolor = red,
    citecolor = blue,
    filecolor = red,
    urlcolor = red
}

% Raccourcis diverses:
\newcommand{\nwc}{\newcommand}
\nwc{\dsp}{\displaystyle}
\nwc{\bge}{\begin{equation}}\nwc{\ene}{\end{equation}}
\nwc{\bgar}{\begin{array}}\nwc{\enar}{\end{array}}
\nwc{\bgit}{\begin{itemize}}\nwc{\enit}{\end{itemize}}

\nwc{\la}{\left\{}\nwc{\ra}{\right\}}
\nwc{\lp}{\left(}\nwc{\rp}{\right)}
\nwc{\lb}{\left[}\nwc{\rb}{\right]}

\nwc{\ul}{\underline}

\nwc{\bgsk}{\bigskip}
\nwc{\vsp}{\vspace{0.1cm}}
\nwc{\vspd}{\vspace{0.2cm}}
\nwc{\vspt}{\vspace{0.3cm}}
\nwc{\vspq}{\vspace{0.4cm}}

\def\N{{\rm I\kern-.1567em N}}
\def\R{{\rm I\kern-.1567em R}}
\def\Z{{\sf Z\kern-4.5pt Z}}

\nwc{\tm}{\times}

\newcounter{nex}\setcounter{nex}{0}
\newenvironment{EX}{%
\stepcounter{nex}
\bgsk{\large {\bf Exercice }\arabic{nex}}\hspace{0.5cm}
}{}

\nwc{\bgex}{\begin{EX}}\nwc{\enex}{\end{EX}}


\setlength{\topmargin}{0cm}
\setlength{\oddsidemargin}{10cm}
\setlength{\evensidemargin}{10cm}
\setlength{\textwidth}{21.cm}
\setlength{\textheight}{25cm}
\setlength{\columnsep}{30pt}
\setlength{\columnseprule}{1pt}
\setlength{\headsep}{0in}
\setlength{\parskip}{0ex}
\setlength{\parindent}{0mm}
\setlength{\footskip}{1.5cm}
\setlength{\voffset}{0cm}


\newcommand{\ct}{\centerline}
\newcommand{\ctbf}[1]{\ct{\bf #1}}

\renewcommand{\no}{\noindent}

\nwc{\bgfg}{\begin{figure}}\nwc{\enfg}{\end{figure}}
  \nwc{\epsx}{\epsfxsize}\nwc{\epsy}{\epsfysize}
\nwc{\bgmp}{\begin{minipage}}\nwc{\enmp}{\end{minipage}}

\newenvironment{theoreme}{\paragraph{Théorème:} \it}{}
\nwc{\bgth}{\begin{theoreme}}\nwc{\enth}{\end{theoreme}}

\newenvironment{Notation}{\paragraph{\ulg{Notation:}} \it}{}
\nwc{\bgnot}{\begin{Notation}}\nwc{\enot}{\end{Notation}}

\newenvironment{lemme}{\paragraph{Lemme:} \it}{}
\nwc{\bglem}{\begin{lemme}}\nwc{\enlem}{\end{lemme}}

\newtheorem{corol}{Corollaire}

\renewcommand{\thesection}{\Roman{section}}
\renewcommand{\thesubsection}{\arabic{subsection}}
\renewcommand{\thesubsubsection}{\alph{subsubsection})}

\nwc{\ulr}[1]{\textcolor{red}{\underline{\textcolor{black}{#1}}}}
\nwc{\ulb}[1]{\textcolor{blue}{\underline{\textcolor{black}{#1}}}}
\nwc{\ulg}[1]{\textcolor{green}{\underline{\textcolor{black}{#1}}}}

\nwc{\sectionc}[1]{\section{\ulr{#1}}}
\nwc{\subsectionc}[1]{\subsection{\ulr{#1}}}
\nwc{\subsubsectionc}[1]{\subsubsection{\ulr{#1}}}

\newenvironment{definitioncolor}{\paragraph{\ulb{Définition:}} \it}{}
\nwc{\bgdefc}{\begin{definitioncolor}}
\nwc{\enefc}{\end{definitioncolor}}


\newenvironment{propcolor}{\paragraph{\ulr{Propriété:}} \it}{}
\nwc{\bgpropc}{\begin{propcolor}}
\nwc{\enpropc}{\end{propcolor}}


\newlength{\colu}
\newlength{\cold}
\newlength{\colt}

\setlength{\colu}{8cm}
\setlength{\cold}{0.9\textwidth}
\setlength{\colt}{6cm}

\nwc{\deftitle}{Définition}
\newlength{\ldef}\settowidth{\ldef}{\deftitle:}
\nwc{\bgdef}[1]{\paragraph{\ulb{\deftitle:}} 
  \begin{minipage}[t]{\cold-\ldef-2em}{\it #1}
  \end{minipage}
}

\nwc{\proptitle}{Propriété}
\newlength{\lprop}\settowidth{\lprop}{\proptitle:}
\nwc{\bgprop}[1]{\paragraph{\ulb{\proptitle:}} 
  \begin{minipage}[t]{\cold-\ldef-2em}{\it #1}
  \end{minipage}
}


\textwidth=18cm
\textheight=24cm
\topmargin=-1cm
\oddsidemargin=-1cm

% Bandeau en bas de page
\newcommand{\TITLE}{Nombres, ordre et intervalles}
\author{Y. Morel}
\date{}

\usepackage{fancyhdr}
\pagestyle{fancyplain}
\setlength{\headheight}{0cm}
\renewcommand{\headrulewidth}{0pt}
\renewcommand{\footrulewidth}{0.1pt}
\lhead{}\chead{}\rhead{}
\lfoot{Y. Morel - \href{https://xymaths.fr/Lycee/2nde/Mathematiques-2nde.php}{xymaths.fr - 2nde}}
\cfoot{}
\rfoot{\TITLE\ - $2^{\text{nde}}$ - \thepage/\pageref{LastPage}}
\footskip=2cm
%%%%%%%%%%%%%%%%%%%%%%%%%%%%%%%%%%%%%%%%%%%%%%%%%%%%%%%%%%%%%%%%%%%%%
\begin{document}

\hfill{\LARGE \bf \TITLE}
\hfill $2^{\mbox{\scriptsize{nde}}}$


\setcounter{section}{0}
\sectionc{Ordre}
 
 \bgdef{
   Salut comment tu vas a et b deux nombres réels. 
   On dit que 
    \bgit 
      \item[$\bullet$] a est inférieur à b, et on note $a<b$ si
	$a-b<0$. 
      \item[$\bullet$] a est supérieur à b, et on note $a>b$ si
      $a-b>0$.
    \enit
 }

 \ul{Ex:} $2<3$ car $2-3=-1<0$\,;\ $-5>-6$ car $-5-(-6)=1>0$
 
 \bgprop{
   Si $a<b$ et $b<c$, alors $a<c$. 
 }

 \ul{Ex:} $-1<0$ et $0<3$, donc $-1<3$\,;
 %\hspace{0.8cm} 
 $\sqrt{2}<2$ et $\pi>2$, donc $\sqrt{2}<\pi$\,;

 \vspd
 \hspace{0.8cm} $\dsp\frac{2}{3}<1$ et $\dsp\frac{12}{11}>1$, donc
            $\dsp\frac{2}{3}<\frac{12}{11}$ 
 

 \bgprop{
   Si $a>0$, $b>0$ et $a>b$ alors $a^2>b^2$ et $\sqrt{a}>\sqrt{b}$ 
   et $\dsp\frac{1}{b}<\frac{1}{a}$.
 }
 
 \vspd
 \ul{Démonstration:}  
   $\bullet$
     $a^2-b^2=\underbrace{\lp a+b\rp}_{>0}\underbrace{\lp a-b\rp}_{>0}>0$, 
     d'où, $a^2>b^2$. 

     \vspd
   $\bullet$
     $\dsp\sqrt{a}-\sqrt{b}=\frac{\lp\sqrt{a}-\sqrt{b}\rp\lp\sqrt{a}+\sqrt{b}\rp}{\sqrt{a}+\sqrt{b}}=\frac{a-b}{\sqrt{a}+\sqrt{b}}>0$ 
     \hspace{0.3cm}$\bullet$ $\dsp\frac{1}{b}-\frac{1}{a}=\frac{b-a}{ab}<0$.
  %\enit
  
  \vspt
  \ul{Exemples et contre-exemples:}

  $\bullet$ $3<5$ donc $3^2<5^2$ \hspace{2.61cm}
  $\bullet$ $2<\pi$ donc $\sqrt{2}<\sqrt{\pi}$ et 
    $\dsp\frac{1}{\pi}<\frac{1}{2}$

  $\bullet$ $-5<-3$ mais $(-5)^2>(-3)^2$ \hspace{1cm} 
  $\bullet$ $-3<2$ mais $-\frac{1}{3}<\frac{1}{2}$


  \bgprop{
    Si $a<b$ et $c<d$ alors $a+c<b+d$. 
  }

  \bgprop{
    Si $a<b$ et $c>0$ alors $ac<bc$.
    Si $a<b$ et $c<0$ alors $ac>bc$.
  }
  
  \vspd
  \ul{Ex:} $2<3$ alors $2\tm 2<3\tm 2$ mais $-2>-3$

  \bgprop{
    Si a, b, c et d sont des nombres \ul{positifs} et si $a<b$ et
    $c<d$ alors $ac<bd$.
  }

  \vspd
  \ul{Exemples et contre-exemples:}\vspd

  $\bullet$ $3<5$ et $\pi<4$, donc $3\pi<20$ \hspace{1cm}
  $\bullet$ $-4<5$ et $-2<-1$, mais $8>-5$  
  
 \bgprop{
   Si $0<a<1$ alors $a^3<a^2<a$. 
   Si $a>1$ alors, $a^3>a^2>a$. 
 }


 \sectionc{Intervalles de $\R$}

 \bgdef{
   Soit a et b deux nombres réels tesl que $a<b$. 
   L'ensemble des nombres réels x vérifiant la double inégalité 
   $a\leq x\leq b$ est appelé intervalle fermé de $\R$. 
   On le note $[a,b]$. 
   
   Les nombres a et b sont les bornes de l'intervalle $[a,b]$, et b-a
   son amplitude (ou longueur). 
 }

% \vspd
% \ct{\ulb{Intervalles de $\R$} }

 \vspd
 \setlength{\unitlength}{1cm}
 \hspace{1.5cm}
 \begin{tabular}{|c|c|c|} \hline
   Intervalle & Encadrement & Représentation sur une droite\\\hline
   $[a,b]$ & $a\leq x\leq b$ & 
   \makebox[5cm]{
     \put(-2,0){\vector(1,0){4}}
     \put(-1,0){\textcolor{red}{\line(1,0){2}}}
     \put(-1.1,0.3){a}\put(1.,0.3){b}
     \put(-1.1,-.1){$[$}\put(1.,-.1){$]$}
   } \\\hline
   $]a,b[$ & $a< x< b$ & 
   \makebox[5cm]{
     \put(-2,0){\vector(1,0){4}}
     \put(-1,0){\textcolor{red}{\line(1,0){2}}}
     \put(-1.1,0.3){a}\put(1.,0.3){b}
     \put(-1.1,-.1){$]$}\put(1.,-.1){$[$}
   } \\\hline
   $]a,b]$ &  & \\\hline
   $[a,b[$ &  & \\\hline
   $[a,+\infty[$ & $a\leq x<\infty$ & 
	       \makebox[5cm]{
		 \put(-2,0){\vector(1,0){4}}
		 \put(-1,0){\textcolor{red}{\vector(1,0){3}}}
		 \put(-1.1,0.3){a}
		 \put(-1.1,-.1){$[$}
	       } \\\hline

   $]a,+\infty[$ &  & \\\hline
   $]-\infty,b]$ &  & \\\hline
   $]-\infty,b[$ &  & \\\hline


 \end{tabular}

 
 \sectionc{Valeur absolue et distance}
 
 \bgdef{
   Pour $x$ un nombre réel, on appelle valeur absolue de x, notée
   $|x|$, le nombre \ulr{positif}: 
    \[ |x| = \la\bgar{l} x\ \mbox{ si, } x\geq0 \\
                         -x\ \mbox{ si, } x\leq0 \enar\right.\]
 }

 \ul{Ex:} $|2,35|=2,35$\,;\ $|-124,36|=124,36$\,;\
     $|-\sqrt{197}|=\sqrt{197}$\,;\ 

     Si $x$ est un nombre réel, $|x-2|=\dots$.

     
 \bgdef{
   On appelle distance entre deux nombres réels x et y, notée dist(x,y),
   celui des deux nombres $x-y$ ou $y-x$ qui est positif. 
 }
 \vspace{-0.3cm}
 \bgprop{
   La distance entre les nombres réels x et y est dist(x,y)$=|x-y|=|y-x|$.
 }

 \ul{Ex:} La distance entre 

 $\bullet$ $x=2$ et $y=3$ est $|3-2|=|2-3|=1$
 
 $\bullet$ $x=-2$ et $y=6$ $\dots$

 $\bullet$ $x=-16$ et $y=-4$

 
 \ulg{Remarque:} La valeur absolue de x est la distance entre x est
 0. 


 \bgprop{
   Soient c un nombre réel et r un nombre réel positif; les quatres
   propositions suivantes sont équivalentes: 
   
   \setlength{\unitlength}{1cm}
   \bgit
     \item[$\bullet$] la distance de x à c est inférieure ou égale à r;
     \item[$\bullet$] $x\in [c-r;c+r]$; 
     \item[$\bullet$] $|x-c|\leq r$; 
       \put(2,0){\vector(1,0){6}}
       \put(3,0){\textcolor{red}{\line(1,0){4}}}
       \put(4,0.4){\vector(1,0){1}}\put(4,0.4){\vector(-1,0){1}}
       \put(6,0.4){\vector(1,0){1}}\put(6,0.4){\vector(-1,0){1}}
       \put(4,0.6){$r$}\put(6,0.6){$r$}
       \put(5,-.1){$|$}\put(5,-0.4){$c$}
       \put(3,-.1){\textcolor{red}{$[$}}\put(2.7,-.5){$c-r$}
       \put(7,-.1){\textcolor{red}{$]$}}\put(6.7,-.5){$c+r$}

     \item[$\bullet$] $c-r\leq x\leq c+r$;
   \enit
   }

 \ul{Ex:}
 $\bullet$ $|x-2|\leq 3$ $\cdots$
 
 $\bullet$ $x\in [-3,5]$ s'écrit de façon équivalente $|x-1|\leq 4$
 car $1=\frac{-3+5}{2}$ et $4=\frac{5-(-3)}{2}=\frac{\mbox{dist}(x,y)}{2}$.




 \sectionc{Valeur approchée d'un nombre réel}

 \bgdef{
   On appelle valeur approchée d'un nombre réel x à la précision e (ou
   à ``e près'') tout nombre réel a tel que $|x-a|\leq e$
 }

 \setlength{\unitlength}{1cm}
 \makebox[10cm]{
   \put(-2,0){\vector(1,0){4}}
   \put(-1,0){\textcolor{red}{\line(1,0){2}}}
   \put(-1.3,0.3){$a-e$}\put(0.7,0.3){$a+e$}
   \put(-1.1,-.1){$[$}\put(1.,-.1){$]$}
   \put(0,-.1){$|$}\put(-.1,0.3){$a$}
   \put(-.5,-.4){$x$}
 }

 \ul{Ex:} $\bullet$ $1,4$ est une valeur approchée de $\sqrt{2}$ à
 $0,1$ près car $|\sqrt{2}-1,2|\leq 0,1$

 $\bullet$ $3,1$ et $3,12$ sont deux valeurs approchées de $\pi$ à
 $0,1$ près. 

 \ulr{Remarque:} Les calculatrices et ordinateurs calculent et
 utilisent des valeurs approchées des nombres réels à $\cdots$ près. 

 Ex: $\pi\sim\cdots$

\label{LastPage}
\end{document}
